\documentclass{article}
\usepackage{graphicx}
\usepackage{tikz}
\usepackage{float}

\begin{document}

\title{Design of Trusted User Interface emulator for Open-TEE project}
\author{Mika Tammi}

\maketitle

\begin{abstract}
This document describes the architecture and design of Trusted User Interface
emulator for Open-TEE project.
\end{abstract}

\section{Introduction}

Trusted User Interface specification is part of GlobalPlatform Device
Specifications. Intention of this work is to provide implementation of Trusted
User Interface API to Open-TEE project.

Section 2 briefly describes the architecture from top level.

\section{Architecture}

Open-TEE will be extended to support the GlobalPlatform Trusted User Interface
API. Essential part of the Open-TEE architecture is the manager process, which
provides services to Trusted Applications through socket-based IPC mechanism.
Trusted Applications are running in separate processes.

Virtual Display software will be created for this project. The Virtual Display
will act as a display screen for the Trusted User Interface API functionality
in the emulation process. The display will be configurable so that it is
possible to emulate various different kinds of screens that could possibly
exist in today's mobile devices and different aspects of API functionality.
Different displays have different properties so it is important to be able to
test various kinds of displays during development.

% Include architecture block diagram figure
% fig:architecture_block_diagram
\begin{figure}

% Define layers
\pgfdeclarelayer{background}
\pgfdeclarelayer{foreground}
\pgfsetlayers{background,main,foreground}

% Define block styles used later
\tikzstyle{process} = [draw, text width=10em, fill=red!20,
    text centered, minimum height=3em, rounded corners]
\tikzstyle{ta} = [draw, text width=6em, fill=red!20,
    text centered, minimum height=3em, rounded corners]

% Define distances for bordering
\def\blockdist{2.3}
\def\edgedist{2.5}

\begin{tikzpicture}
    \node (manager) [process] {Manager Process};
    \path (manager.south)+(-3.0,-1.0) node (ta1) [ta] {Trusted Application 1};
    \path (manager.south)+(0.0,-1.0) node (ta2) [ta] {Trusted Application 2};
    \path (manager.south)+(3.0,-1.0) node (ta3) [ta] {Trusted Application 3};

    \path [draw, -] (ta1.north) -- node [above] {} (manager.south);
    \path [draw, -] (ta2.north) -- node [above] {} (manager.south);
    \path [draw, -] (ta3.north) -- node [above] {} (manager.south);

    \path (manager.south) +(0,-\blockdist) node (teeemu) {TEE Emulator};

    \begin{pgfonlayer}{background}
        % a and b are corner nodes for the rectangle later
        \path (ta1.west |- manager.north)+(-0.5,0.3) node (a) {};
        \path (ta3.east |- teeemu.east)+(+0.5,-0.5) node (b) {};

        % Draw a nice yellow rectangle on background layer
        \path[fill=yellow!20,rounded corners, draw=black!50, dashed]
            (a) rectangle (b);
    \end{pgfonlayer}

    \path (manager.west)+(-5.0,0.0) node (virtuald) [process] {Virtual Display};
    \path (virtuald.north)+(0.0,2.0) node (ca) [ta] {Client Application};

    \path [draw, -] (virtuald.east) -- node [above] {Trusted UI Socket} (manager.west);
    \path [draw, -] (ca.east) -- node [above] {Client Socket} (manager.north);
\end{tikzpicture}
\caption{Block diagram for top-level architecture of Trusted User Interface API
         implementation in Open-TEE}
\label{fig:architecture_block_diagram}
\end{figure}


The manager process of the Open-TEE emulator will be extended so that it
provides communication between the Trusted Application, that wants to use
functionality provided by the Trusted User Interface API, and the Virtual
Display software, acting as a kind of a router. Communication to Virtual
Display is provided through socket-based interface. Manager will create the
socket and listen to it, and the Virtual Display will then connect to that.

Figure \ref{fig:architecture_block_diagram} shows the top-level architecture of
Open-TEE. Labeled boxes represent different processes or applications and lines
between boxes represent socket connections. Manager process acts as a router
for different processes, routing inter-process call messages from process to
another. The reason for this kind of design is that the Trusted Applications
are isolated from each other. If one Trusted Application crashes or starts to
create problems, it doesn't necessarily bring the whole Open-TEE environment
down.

\section{Design}

The Trusted User Interface API implementation consists of few components:
\begin{itemize}
    \item{Virtual Display software providing the screen for the
          Trusted User Interface}
    \item{Manager process extensions: Manager process will be extended so that
          it listens to virtual display socket, and routes the IPC messages
          from/to Trusted Application using Trusted User Interface API}
    \item{Trusted User Interface API that is used to interface towards
          Trusted Applications}
\end{itemize}

\subsection{Virtual Display}

Desktop version of Virtual Display will be implemented with Qt Framework.
Virtual Display will allow different screens to be configured so that it is
possible to test how Trusted Application would show the User Interface in
variety of scenarios.

In the Virtual Display, the things that should be customizable are:
\begin{itemize}
    \item{Font}
    \item{Supported orientations: portrait or landscape or both}
    \item{Duration for session timeout}
    \item{Display size and reported pixel-per-inch resolution}
    \item{Maximum number of entry fields}
    \item{Default color for label}
    \item{Can image and text be customized for each individual button type}
\end{itemize}

User interface for the Virtual Display will be quite simple. There will be a
main window that shows the context of the Virtual Display, and a preferences
dialog for changing the settings mentioned above.

\subsection{API for Trusted Applications}

The Trusted User Interface API will be provided as the GlobalPlatform Trusted
User Inter API -document specifies it. The document specifies a C API to be
used by Trusted Application.

Trusted User Interface API provides 5 functions for interacting with display.
Layout functionality is mostly implementation defined, with some possibility
for customization.

The functions provided are:
\begin{itemize}
    \item{Function for checking metrics for the given text}
    \item{Function for getting information about the display, such as size, how
          many input fields would fit there, and what is customizable}
    \item{Function for opening the Trusted User Interface session}
    \item{Function for closing the Trusted User Interface session}
    \item{Function for actually displaying the defined layout and getting back
          the user input}
\end{itemize}

\subsection{Protocol}

Protocol between Open-TEE emulator and the Virtual Display consists of 10
different types of messages. There will be request and response message for
each of the function calls. Request message will be sent to Virtual Display
from Trusted Application when function is called. Function is then blocked
until the response is received from Virtual Display.

Basically the request message will contain all the parameters given for Trusted
User Interface API function in serialized form, and the response will then have
all the return data.

\section{Testing}

For testing, a stub implementation of both Virtual Display and Open-TEE Manager
process socket listener will be implemented. This enables easier development of
the individual components since the whole software stack doesn't need to be run
at the same time.

Another aspect for testing is about testing the Trusted Applications. The
Virtual Display stub could also be used to create an automated testing
environment for Trusted Applications that use the Trusted User Interface API.
For example, one could create a test set that automatically tests the Trusted
Application behaviour with various kinds of different display setups.

\section{Summary}

This concludes the design document for Trusted User Interface emulator for
Open-TEE project. Architecture and design was briefly introduced and this
document should reflect to the actual implementation of the Trusted User
Interface API.

\end{document}
